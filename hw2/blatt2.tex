\documentclass[]{eptcs}
\usepackage[utf8]{inputenc}

\usepackage[toc,page]{appendix}

% Defines \FloatBarrier
% The "section" parameter makes floats stay in the section they are defined in.
\usepackage[section]{placeins}

\usepackage{graphicx}
\usepackage{listings}
\usepackage{parcolumns}
\usepackage{amsmath}
\usepackage{url}
\usepackage{breakurl}
\usepackage{todonotes}


\renewcommand{\abstractname}{Abstract}

%TODO: center listings
\lstset{language=C,frame=L,basicstyle=\small\ttfamily,tabsize=2}
\lstset{numbersep=10pt}
\lstset{xleftmargin=5.0ex}
\lstset{numbers=left}
% Captions to the bottom
\lstset{captionpos=b}
\lstset{breaklines=false}

% Not used and has a tendency to break code.
\lstset{mathescape=false}

\newcommand{\ie}{i.~e.}
\newcommand{\Ie}{I.~e.}
\newcommand{\eg}{e.g.}
\newcommand{\Eg}{E.~g.}



\title{Verification of Reactive Systems\\
       Problem Set 2}

\author{Julius Roob
	\email{julius@juliusroob.de}
}

\begin{document}
	\maketitle \newpage
	%\tableofcontents \newpage
	\listoftodos \newpage
	
	\newpage
	\section*{Problem T1.1.}
		\subsection*{a)}
			Simple backtracking on a list containing the path during recursive descend/ascend.
		
		\subsection*{b)}
			Breadth-first search requires the trajectories to be saved alongside the nodes to be visited in the queue.
			Unlike backtracking, this approach will require a significant amount of memory.
	
	\newpage
	\section*{Problem T1.2.}
		Equivalence checks can be reduced to sat queries by the following rule:
		\begin{align*}
			\neg sat(\neg (C_{1} \Leftrightarrow C_{2})) \Leftrightarrow (C1 \Leftrightarrow C2)
		\end{align*}
		
		As for the CNF requirement, every boolean expression can be rewritten to CNF, for example using the library attached for E1.4.:
\begin{lstlisting}
def is_equiv(C1, C2):
	return None == run_minisat(~(EQUIV(C1, C2)).to_cnf())
\end{lstlisting}
	
	
	\newpage
	\section*{Problem T1.3.}
		Assuming the SAT-solver is working correctly, and that Resulution is known to be refutation-complete, the statement holds.
	
	\newpage
	\section*{Problem E1.4.}
		The websudoku solving code is given in the code/test.py file.
		
		Minisat can be used to generate sudokus, it provides (more or less) random assignments when given the sudoku rules.
		These can be turned into sudoku puzzles by removing entries.
		Uniqueness can be checked by:
		
\begin{lstlisting}
	rules = sudok_rules_simple().to_cnf()
def is_unique(sudoku):
	assignment = reduce(AND, run_minisat(sudoku.constraints().to_cnf() & rules))
	return None == run_minisat((sudoku.constraints() & ~assignment).to_cnf())
\end{lstlisting}
	
	\newpage
	\section*{Paper Reading}
		
\end{document}
